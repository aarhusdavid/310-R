\PassOptionsToPackage{unicode=true}{hyperref} % options for packages loaded elsewhere
\PassOptionsToPackage{hyphens}{url}
%
\documentclass[]{article}
\usepackage{lmodern}
\usepackage{amssymb,amsmath}
\usepackage{ifxetex,ifluatex}
\usepackage{fixltx2e} % provides \textsubscript
\ifnum 0\ifxetex 1\fi\ifluatex 1\fi=0 % if pdftex
  \usepackage[T1]{fontenc}
  \usepackage[utf8]{inputenc}
  \usepackage{textcomp} % provides euro and other symbols
\else % if luatex or xelatex
  \usepackage{unicode-math}
  \defaultfontfeatures{Ligatures=TeX,Scale=MatchLowercase}
\fi
% use upquote if available, for straight quotes in verbatim environments
\IfFileExists{upquote.sty}{\usepackage{upquote}}{}
% use microtype if available
\IfFileExists{microtype.sty}{%
\usepackage[]{microtype}
\UseMicrotypeSet[protrusion]{basicmath} % disable protrusion for tt fonts
}{}
\IfFileExists{parskip.sty}{%
\usepackage{parskip}
}{% else
\setlength{\parindent}{0pt}
\setlength{\parskip}{6pt plus 2pt minus 1pt}
}
\usepackage{hyperref}
\hypersetup{
            pdftitle={HomeWork2},
            pdfauthor={David Aarhus},
            pdfborder={0 0 0},
            breaklinks=true}
\urlstyle{same}  % don't use monospace font for urls
\usepackage[margin=1in]{geometry}
\usepackage{graphicx,grffile}
\makeatletter
\def\maxwidth{\ifdim\Gin@nat@width>\linewidth\linewidth\else\Gin@nat@width\fi}
\def\maxheight{\ifdim\Gin@nat@height>\textheight\textheight\else\Gin@nat@height\fi}
\makeatother
% Scale images if necessary, so that they will not overflow the page
% margins by default, and it is still possible to overwrite the defaults
% using explicit options in \includegraphics[width, height, ...]{}
\setkeys{Gin}{width=\maxwidth,height=\maxheight,keepaspectratio}
\setlength{\emergencystretch}{3em}  % prevent overfull lines
\providecommand{\tightlist}{%
  \setlength{\itemsep}{0pt}\setlength{\parskip}{0pt}}
\setcounter{secnumdepth}{0}
% Redefines (sub)paragraphs to behave more like sections
\ifx\paragraph\undefined\else
\let\oldparagraph\paragraph
\renewcommand{\paragraph}[1]{\oldparagraph{#1}\mbox{}}
\fi
\ifx\subparagraph\undefined\else
\let\oldsubparagraph\subparagraph
\renewcommand{\subparagraph}[1]{\oldsubparagraph{#1}\mbox{}}
\fi

% set default figure placement to htbp
\makeatletter
\def\fps@figure{htbp}
\makeatother


\title{HomeWork2}
\author{David Aarhus}
\date{2/19/2020}

\begin{document}
\maketitle

\hypertarget{r-markdown}{%
\subsection{R Markdown}\label{r-markdown}}

This is an R Markdown document. Markdown is a simple formatting syntax
for authoring HTML, PDF, and MS Word documents. For more details on
using R Markdown see \url{http://rmarkdown.rstudio.com}.

When you click the \textbf{Knit} button a document will be generated
that includes both content as well as the output of any embedded R code
chunks within the document. You can embed an R code chunk like this:

rm(list = ls()) \#removing all variables

\#Question 1a college \textless{}-
read.csv(``/Users/DavidAarhus/Documents/310 R/Datasets/college.csv'')

\#Question 1b rownames(college) = college{[},1{]} \#established the
rownames for each college head(college)

college \textless{}- college{[},-1{]} \#deletes the 1st column in the
dataset since it \#is being used to identify the rows head(college)

\#Question 1c summary(college) \#function to produce a numerical summary
of \#the variables in the data set

\#Question 1d pairs(college{[} , 1:10{]}) \#a scatterplot matrix of the
first ten columns \#or variables of the \#data.

\#Question 1e plot(college\(Private, college\)Outstate) \#function to
produce side-by-side boxplots of \#Outstate versus Private

\#Question 1f Elite \textless{}- rep(``No'',nrow(college)) \# Creates
new Elite variable Elite{[}college\$Top10perc \textgreater{}50{]}
\textless{}- ``Yes'' \# whether or not the proportion of students coming
from the Elite \textless{}- as.factor(Elite) \# top 10\% of their high
school classes exceeds 50\%. college \textless{}- data.frame(college,
Elite) \# rerunning the college dataframe

summary(college) \#gives dataframes statistics
plot(college\(Elite, college\)Outstate) \#produces boxplot for Outstate
and Elite

\#Question 1g par(mfrow=c(2,2))
hist(college\(Top10perc) hist(college\)Enroll)
hist(college\(Top25perc) hist(college\)Books)

\#Question 1h par(mfrow=c(1,1)) plot(college\(Apps, college\)Enroll)
\#as applications for colleges rise so does the Enroll
plot(college\(Apps, college\)Accept) \#more apps equal more Acceptions
hist(college\(Enroll) #most schools enroll below 1000 students hist(college\)Room.Board)
\#This shows a wide range between room and board for universities, with
the average \#coming in around \$4500

\#Question 2a library(``MASS'') \#Now the data set is contained in the
object Boston. head(Boston) \#Read about the data set by ?Boston
nrow(Boston) \#gets number of rows ncol(Boston) \#gets number of columns
Boston \textless{}- data.frame(Boston) \#the columns give information
about the population at Boston \#crime, indus, age, tax, race, etc
colnames(Boston) ?Boston

\#Question 2b library(``corrplot'') \#loads correlation plot
availability corrplot(cor(Boston{[},{]})) \#creates correlation plot
\#proportion of non-retail business acres per town and full-value
property-tax rate per \$10,000. \#highly correlated \#index of
accessibility to radial highways and per capita crime rate by town.
\#highly correlated

\#Question 2c \#yes, the index of accessibility to radical highways is
correlated to the crime rate \#also, full-value property-tax rate per
\$10,000 is correlated to crime rate

\#Question 2d \# The proportion of non-retail business acres per town
have a correlation with high taxes. (\textasciitilde{}0.6 - 0.8) \# Also
there is an interesting correlation between lower status of the
population and high tax rate (\textasciitilde{}0.4 - 0.6) \# The
proportion of non-retail business acres per town also has high
correlation with high tax rates (\textasciitilde{}0.6 - 0.8) \#
Pupil-teacherratios is correlated with tax and rad (\textasciitilde{}0.2
- 0.4)

\#Question 2e sum(Boston{[},4{]}) \#35 uburbs in this data set bound the
Charles river

\#Question 2f median((Boston{[},4{]})) \#median is 0

\#Question 2g min(Boston{[},14{]}) \#takes lowest median value of
owner-occupied homes \# min median: 5 )

(Boston{[}399,{]}) \#lists values of all the predictors for the lowest
median of owner-occupied homes range(Boston{[},1{]}) (Boston{[}399,1{]})
range(Boston{[},2{]}) (Boston{[}399,2{]}) range(Boston{[},3{]})
(Boston{[}399,3{]}) range(Boston{[},4{]}) (Boston{[}399,4{]})
range(Boston{[},5{]}) (Boston{[}399,5{]}) range(Boston{[},6{]})
(Boston{[}399,6{]}) range(Boston{[},7{]}) (Boston{[}399,7{]})
range(Boston{[},8{]}) (Boston{[}399,8{]}) range(Boston{[},9{]})
(Boston{[}399,9{]}) range(Boston{[},10{]}) (Boston{[}399,10{]})
range(Boston{[},11{]}) (Boston{[}399,11{]}) range(Boston{[},12{]})
(Boston{[}399,12{]}) range(Boston{[},13{]}) (Boston{[}399,13{]})
range(Boston{[},14{]}) (Boston{[}399,14{]}) \#functiions above list the
ranges for all the predictors in the dataset \#Then I listed the value
of the indicator with the min median Owner-occupied value \#Almost all
the predictors were higher than most of the other observations except,
zn, dis, and medv \#This row had the highest Bk proportion

\#Questionh
sum(Boston\(rm > 7) #how many of the suburbs average more than seven rooms per dwelling sum(Boston\)rm
\textgreater{} 8) \#how many of the suburbs average more than eight
rooms per dwelling Boston{[} Boston\$rm \textgreater{} 8 , {]} \#list
values for suburbs that average more than eight rooms per dwelling

\#I noticed that all of them have high bk values and low lstat values.

Note that the \texttt{echo\ =\ FALSE} parameter was added to the code
chunk to prevent printing of the R code that generated the plot.

\end{document}
